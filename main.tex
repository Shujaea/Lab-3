%%%%%%%%%%%%%%%%%%%%%%%%%%%%%%%%%%%%%%%%%%%
%%% DOCUMENT PREAMBLE %%%
\documentclass[12pt]{report}
\usepackage[english]{babel}
%\usepackage{natbib}
\usepackage{url}
\usepackage[utf8x]{inputenc}
\usepackage{amsmath}
\usepackage{graphicx}
\graphicspath{{images/}}
\usepackage{parskip}
\usepackage{fancyhdr}
\usepackage{vmargin}
\usepackage{listings}
\usepackage{hyperref}
\usepackage{xcolor}

\definecolor{codegreen}{rgb}{0,0.6,0}
\definecolor{codegray}{rgb}{0.5,0.5,0.5}
\definecolor{codeblue}{rgb}{0,0,0.95}
\definecolor{backcolour}{rgb}{0.95,0.95,0.92}

\lstdefinestyle{mystyle}{
    backgroundcolor=\color{backcolour},   
    commentstyle=\color{codegreen},
    keywordstyle=\color{codeblue},
    numberstyle=\tiny\color{codegray},
    stringstyle=\color{codegreen},
    basicstyle=\ttfamily\footnotesize,
    breakatwhitespace=false,         
    breaklines=true,                 
    captionpos=b,                    
    keepspaces=true,                 
    numbers=left,                    
    numbersep=5pt,                  
    showspaces=false,                
    showstringspaces=false,
    showtabs=false,                  
    tabsize=2
}
 
\lstset{style=mystyle}

\setmarginsrb{3 cm}{2.5 cm}{3 cm}{2.5 cm}{1 cm}{1.5 cm}{1 cm}{1.5 cm}

\title{Lab3}								
% Title
\author{ Shujaea Aldousari}						
% Author
\date{Sep 23, 2021}
% Date

\makeatletter
\let\thetitle\@title
\let\theauthor\@author
\let\thedate\@date
\makeatother

\pagestyle{fancy}
\fancyhf{}
\rhead{\theauthor}
\lhead{\thetitle}
\cfoot{\thepage}
%%%%%%%%%%%%%%%%%%%%%%%%%%%%%%%%%%%%%%%%%%%%
\begin{document}

%%%%%%%%%%%%%%%%%%%%%%%%%%%%%%%%%%%%%%%%%%%%%%%%%%%%%%%%%%%%%%%%%%%%%%%%%%%%%%%%%%%%%%%%%

\begin{titlepage}
	\centering
    \vspace*{0.5 cm}
   % \includegraphics[scale = 0.075]{bsulogo.png}\\[1.0 cm]	% University Logo
\begin{center}    \textsc{\Large   ECE 351 - Section 52 \# }\\[2.0 cm]	\end{center}% University of Idaho
	\textsc{\Large Lab 3 Report  }\\[0.5 cm]				% Course Code
	\rule{\linewidth}{0.2 mm} \\[0.4 cm]
	{ \huge \bfseries \thetitle}\\
	\rule{\linewidth}{0.2 mm} \\[1.5 cm]
	
	\begin{minipage}{0.4\textwidth}
		\begin{flushleft} \large
		%	\emph{Submitted To:}\\
		%	Name\\
          % Affiliation\\
           %contact info\\
			\end{flushleft}
			\end{minipage}~
			\begin{minipage}{0.4\textwidth}
            
			\begin{flushright} \large
			\emph{Submitted By :} \\
			Shujaea Aldousari  
		\end{flushright}
           
	\end{minipage}\\[2 cm]
	
%	\includegraphics[scale = 0.5]{PICMathLogo.png}
    
    
    
    
	
\end{titlepage}

%%%%%%%%%%%%%%%%%%%%%%%%%%%%%%%%%%%%%%%%%%%%%%%%%%%%%%%%%%%%%%%%%%%%%%%%%%%%%%%%%%%%%%%%%

\tableofcontents
\pagebreak

%%%%%%%%%%%%%%%%%%%%%%%%%%%%%%%%%%%%%%%%%%%%%%%%%%%%%%%%%%%%%%%%%%%%%%%%%%%%%%%%%%%%%%%%%
\renewcommand{\thesection}{\arabic{section}}
\section{Introduction}
 

In lab 3, we were introduced with convolutions. The purpose of this lab is to understand /learn about convolutions .


\section{Part 1}

In part one we were asked to use the previous step and ramp functions from the previous lab, Which has been developed, edited and used in this lab.





\begin{lstlisting}[language=Python]


def step(t):
    y = np.zeros((len(t),1))
    for i in range(len(t)):
        if t[i] >= 0:
            y[i] = 1
        else:

            y[i]= 0

    return y       

steps = 1e-4

t = np.arange(-5, 10 + steps, steps)
y = step(t)

def ramp(t):

    y = np.zeros((len(t),1))

    for i in range(len(t)):

        if t[i]>=0:

            y[i] = t[i]

        else:

            y[i] = 0

    return y      

y = ramp(t)

def f1(t):

    y = step(t- 2) - step(t - 9)

    return y


def f2(t):

    y = np.exp(-t)

    return y

def f3(t):

    y =  ramp(t - 2)*(step(t - 2) - step(t - 3)) + ramp(4 - t)*(step(t - 3) - step(t - 4))

   
    return y

steps = 1e-2

t = np.arange(0, 20 + steps, steps)

plt.figure(figsize = (10, 7))
plt.subplot(3, 1, 1)
plt.plot(t, f1(t))
plt.ylabel('f1(t)')
plt.title('User Defined Function ')
plt.ylim([0, 1.2])
plt.grid()
plt.show()
plt.subplot(3, 1, 2)
plt.plot(t, f2(t))
plt.ylabel('f2(t)')
plt.ylim([0, 1.2])
plt.grid()
plt.show()
plt.subplot(3, 1, 3)
plt.plot(t, f3(t))
plt.ylabel('f3(t)')
plt.ylim([0, 1.2])
plt.grid()
plt.xlabel('t(s)')
plt.show()



\end{lstlisting}

I tried attaching the plots here but it did not work for some reason. I used the following commands with including ' \ ' but it would not work.
includegraphics[]{1.png}
includegraphics[]{2.png}
includegraphics[]{3.png}

\section{Part 2 }

In the second part, we were asked to create a convulsion functions for f1 and f2,f2 and f3,and f1 and f3. During the time I was creating the code, I faced a problem which was the size of f1 and f2 is different for some reason so it is not working.

\begin{lstlisting}[language=Python]
def con(fu1, fu2):
    conf1=len(fu1)
    conf2=len(fu2)
    f1E = np.append(fu1, np.zeros((1, conf2-1)))
    f2E = np.append(fu2, np.zeros((1, conf1-1)))
    Value = np.zeros(f1E.shape)
    for i in range (conf2 + conf1 - 2):
        Value[i] = 0
        for j in range(conf1):
            if(i - j + 1 > 0):
                try:
                    Value[i] += f1E[i]*f2E[i - j + 1]
                except:
                    print(i,j)
    return Value
                  
steps = 1e-2
t = np.arange(0, 20 + steps, steps)
Num = len(t)
tE = np.arange(0, 2*t[Num - 1], steps)

fu1 = f1(t)
fu2 = f2(t)
fu3 = f3(t)


con12 = con(fu1, fu2)*steps
# con12Check = sig.convolve(fu1, fu2)*steps
# size of f1 and f2 is differnt for some reason so it is not working.

plt.figure(figsize = (10, 7))
plt.plot(tE, con12, label = ".fined Conv")
# plt.plot(tE, con12Check, '-', label = 'Built in Conv')
plt.ylim([0, 1.2])
plt.grid()
plt.legend()
plt.xlabel('t(s)')
plt.ylabel('fun1(t) * fun2(t)')
plt.title('Conv of fun1 , fun2')
plt.show()

con23 = con(fu2, fu3)*steps
# con23Check = sig.convolve(fu2, fu3)*steps

plt.figure(figsize = (10, 7))
plt.plot(tE, con23, label = 'User defined Conv')
# plt.plot(tE, con23Check, '-', label = 'Built in Conv')
plt.ylim([0, 0.2])
plt.grid()
plt.legend()
plt.xlabel('t [s]')
plt.ylabel('f_2(t) * f_3(t)')
plt.title('Convolution of f_2 and f_3')
plt.show()

con13 = con(fu1, fu3)*steps
con13Check = sig.convolve(fu1, fu3)*steps

plt.figure(figsize = (10, 7))
plt.plot(tE, con13, label = 'User defined Conv')
plt.plot(tE, con13Check, '-', label = 'Built in Conv')
plt.ylim([0, 1.2])
plt.grid()
plt.legend()
plt.xlabel('t [s]')
plt.ylabel('f_1(t) * f_3(t)')
plt.title('Convolution of f_1 and f_3')
plt.show()

\end{lstlisting}

I tried attaching the plots here but it did not work for some reason. I used the following commands with including ' \ ' but it would not work.
includegraphics[]{4.png}
includegraphics[]{5.png}
includegraphics[]{6.png}




\section{Questions}

1. Did you work alone or with classmates on this lab? If you collaborated to get to the solution,
what did that process look like?
2. What was the most difficult part of this lab for you, and what did your problem-solving
process look like?
2
3. Did you approach writing the code with analytical or graphical convolution in mind? Why
did you chose this approach?
4. Leave any feedback on the clarity of lab tasks, expectations, and deliverables.

1. I worked with my classmates in how we can create the code for the convolution.
2. The most difficult part was creating the code for f1 and f2,f2 and f3, f1 and f3.
3.



\section{conclusion}


Overall, the main idea of this lab to introduce new things such as convolution. Also, sometimes there is always an easier way for creating codes, which depends on the ability/skills that can be developed during labs.

\newpage



\end{document}

%This template was created by Roza Aceska.
